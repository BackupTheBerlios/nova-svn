% Created by Andrew Nurse, OpenArrow Software
% This document is licensed under the Common Public License, read it at: http://www.opensource.org/licenses/cpl1.0.php, or send e-mail to openarrow@gmail.com for a copy of the license.

% Labels and References
%  This document uses a structured format to labels and references. All 
%  labels take the form [type]:[refpath] where type is "SEC" (for 
%  references to sections), "GLOB" (for global references), or some other 
%  short all-caps text ("FIG" for figures, "EQN" for equations, etc.). The  
%  refpath is a concatination of the refpath for the
%  "parent item" with a ':' and an identifier for the  current item. For 
%  example, the Introduction section's label is "SEC:Intro" and the label 
%  for the Work In Progress sub-section is "SEC:Intro:WIP." 
%  Because of cross-references both above and below the label which defines 
%  them, this input file should be run through a LaTeX processor twice 
%  (unless the processor automatically performs 2 passes to resolve 
%  cross-references).

\documentclass[a4paper,12pt]{article} 
\usepackage{listings}
\usepackage{color}

% Prepares table contents
\newenvironment{tblcont}[2]{\begin{center}\begin{tabular}{#1}\hline #2\\
	\hline}{\hline\end{tabular}\end{center}}

% Code Listing settings, all listings should have the following key,val pairs in their lstlisting environment begin tag:
% float
% caption = [caption]
% label = [label]
% This ensures the listing floats and that 
\lstset{basicstyle=\tt\small, 
escapeinside={/@}{@/},language=CSharp,frame=leftline,
	keywordstyle=\color{blue}\bfseries,commentstyle=\it,captionpos=b,
	showstringspaces=false}
	
\renewcommand{\thetable}{\arabic{section}.\arabic{table}}
\renewcommand{\thelstlisting}{\arabic{section}.\arabic{lstlisting}}

\begin{document}

\title{Nova - Component-Oriented Application Development}
\author{Andrew Nurse, openarrow@gmail.com}
\date{February 9, 2005}
\maketitle

% Outstanding References
% NEVER - DO NOT RESOLVE!
% These references take the following format
% [RefId] - [Section Title]
% GLOB:COM - Nova vs. The Component Object Model
% GLOB:AppDomains - Application Domains
% GLOB:AppStart - Application Startup Procedure
% GLOB:Contracts - Messaging Contracts
% GLOB:Disco - Discovery Servers
% GLOB:Remoting - Remoting
% GLOB:ComServ - Component Servers
% GLOB:Protocols - Transport Protocols
% GLOB:Naming - Component Naming

%TODO: Finish Introduction
\section{Introduction}
\label{SEC:Intro}

\subsection{Work In Progress}
\label{SEC:Intro:WIP}
% Note: \ref{NEVER} is used to duplicate exactly the "??" that LaTeX outputs whenever a section reference could not be found. Do not use this text as a label!
This document is a work in progress. Anytime you see ``\ref{NEVER}'' as a section reference, it refers to a section which does not yet exist. Please contact the author at ``openarrow@gmail.com'' if you would like to contribute or help in this project. Please report errors in this document to e-mail to the address above. 

This document was created from a \LaTeX 2e input file and can be converted into DVI, PS, PDF, or some other format using an appropriate \LaTeX\ processor. 

\subsection{Assumptions}
\label{SEC:Intro:Assumptions}
\label{GLOB:Assumptions}
This document assumes you have some idea of how the CLR and Microsoft's implementation of it, in particular, work. You should also have some knowledge of the C\# language as most of the code listings in this document are written in that language.

\subsection{Code Listings}
\label{SEC:Intro:Listings}
Throughout this document, sample code will be presented in the form of code listings. These listings are separated out from the document and coloured as they would be in most standard programmer text editors. All listings are in C\# unless otherwise specified. See Listing \ref{LST:Intro:Sample} for a sample listing.

\begin{lstlisting}[float,caption={A Sample Listing},label={LST:Intro:Sample}]
public void SampleMethod()
{
	// This is a sample listing
	Console.WriteLine("Sample Listing");
}
\end{lstlisting}

\section{Overview}
\label{SEC:OVR}

\subsection{What is a Component?}
\label{SEC:OVR:Comp}
The word ``component'' has many meanings in software development. It can be an abstract concept of a single encapsulated entity that performs a certain task. Or it can be a more concrete concept such as the idea of a COM Component (see Section \ref{GLOB:COM}, ``Nova vs. The Component Object Model). Nova is an implementation of the eXtensible Application Development System (XADS, pronounced `zads') which provides a specification for ``Component-Oriented Development.'' COD is a further development of the concepts provided by Object-Oriented Programming (OOP) in that a component, as defined by the concepts of COD, makes use of multiple objects in its implementation. However, a component is very different from an object, and Table \ref{TAB:CvO} summarizes these differences.

\begin{table}
\label{TAB:CvO}
\begin{tblcont}{| l | l |}{\textbf{Component} & \textbf{Object}}
Contains multiple objects & Is one object\\
\hline
Designed to perform a small set of tasks & Represents an entity\\
\hline
Designed for portability & Tied to an application\\
\hline
Designed for transparent remoting & Remoting is implemented by other systems\\
\end{tblcont}
\caption{Differences between Components and Objects}
\end{table}


\subsection{What is an Application?}
\label{SEC:OVR:App}
Like a component, the term ``application'' also has many meanings. Usually it refers to a program that a user can execute to perform a certain task, but this is only a loose definition. Nova, and XADS, defines an application as a particular set of interactions between components as coordinated by an ``Application Component.'' What does this mean? It means that an applications is simply a group of components working together, and this cooperation is managed by a component that is specific to the application. If you are familiar with .Net, and specifically C\# (see Section \ref{GLOB:Assumptions}, ``Assumptions'') this Application Component is similar to the entry point to an application. In effect, the Application Component is loaded by Nova and sent a ``Main'' message (see Section \ref{GLOB:Messaging}, ``Messaging'') which is equivalent to calling the entry point method of a C\# application. The component receives this message and begins sending messages to other components to get them working. Section \ref{GLOB:AppStart}, ``Application Startup Procedure'' details this process.

\subsection{Why should I care?}
\label{SEC:OVR:Care}
So, now you know what Nova considers a component and what it considers an application, but how does this affect you and your development process? This question is easy to answer with one simple statement: ``Components are designed to be as portable as possible.'' This means that if you develop an application, you must try to develop all your components so that they are usable by any other application (with your permission of course, you could even charge for it) and in turn, your application can use other components, even though they are not developed for your application. 

	As an example, consider an archive management application (for managing zip, tar, etc.\ archives), this application has user interface components and management components to manage the interface. Now, that application needs a library to manage its main archive format (say FSA, the Freakin' Small Archive, a hypothetical format), you could write a component for that format and then sell it (or give it away free, or any number of other licensing options) and others could create FSA files easily. Also, you could acquire a ZIP file component and incorporate it into your application, allowing you to manage ZIP files as well. As a final step, you could use Messaging Contracts (see Section \ref{GLOB:Contracts}, ``Messaging Contracts'') so that you can have your application, at runtime, search for archive format components and download them on the fly through Discovery Servers (see Section \ref{GLOB:Disco}, ``Discovery Servers'').
	
	Clearly, Nova provides a great deal of advantages, but there are some costs. Component messaging is not as fast as object messaging (in most cases), because Object messaging is usually implemented as simple function calls. However, component messages are queued and prioritized and may be required to pass between multiple machines (see Section \ref{GLOB:Remoting}, ``Remoting''). Though contracts and disco servers are extremely useful, they usually require some sort of central registry to ensure two contracts do not perform the same task (causing incompatibility between them). Nova requires some infrastructure on the end user's computer in the form of Component Servers (see Section \ref{GLOB:ComServ}, ``Component Servers''). However, in many cases, the benefits outweigh the costs and overall Nova improves application development.
	
\section{Architecture}
\label{SEC:ARCH}
This section will provide a description of the architecture of the Nova system.

\subsection{Messaging}
\label{SEC:ARCH:Messaging}
\label{GLOB:Messaging}
One of the primary goals of Nova is to facilitate communication between components on the same machine, or across the world. This is achieved through a messaging system that is consistent even if the message is sent to a computer halfway around the world. At the core of this system is the message structure itself (see Section \ref{SEC:ARCH:Messaging:Structure}, ``Message Structure''), which defines the exact format of a message. Also, a system for message routing (see Section \ref{SEC:ARCH:Messaging:Routing}, ``Message Routing'') allows messages to cross various application boundaries (see Section \ref{GLOB:AppDomains}, ``Application Domains'') seamlessly and without any extra action by the sender. Finally, a standardized serialization system (see Section \ref{SEC:ARCH:Messaging:Serialization}, ``Message Serialization'') is required to ensure messages can be easily sent over various transport protocols (see Section \ref{GLOB:Protocols}, ``Transport Protocols'').

\subsubsection{Message Structure}
\label{SEC:ARCH:Messaging;Structure}
Nova uses messaging to allow components to request that other components perform actions or send them data. This process is very similar to the way object-oriented messaging works in that an OO message consists of an action to be performed (method name) and data to aid in the performing of that action (parameters). These two attributes (action and parameters), along with the component name (see Section \ref{GLOB:Naming}, ``Component Naming'') of the sender and the reciever, make up the main data contained in the message (other information is discussed in later sections). In Nova, the arguments to a message are .Net \lstinline!object! instances which are serialized by various transport protocols as defined in Section \ref{GLOB:Serialization}, ``Message Serialization.''

\end{document}
